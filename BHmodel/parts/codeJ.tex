\begin{widetext}
\section{Code listing: Julia} \label{app:codesP}

The calculation of the ground state of the system can be implemented using DMRG in Julia and the ITensor package \cite{Fishman_2022}, which was convenient for verifying the correctness of the implementation on larger system sizes.
\begin{lstlisting}[language=Python]
using NPZ
using ITensors
using ITensorMPS


function calculate_adagacorr(L, mu, U, J, nsweeps, maxdim, cutoff, bc, nc)
    file_name = "data/J$(round(Int,100*J))_mu$(round(Int,100*mu))_L$(L)_bc$(bc)_nc$(nc)_sweeps$(nsweeps)_dim$(maxdim[1]).npy"
    println("processing $(file_name[6:end-4])")
    
    # Initialize the sites
    sites = siteinds("Boson", L; dim=5, conserve_qns=false)

    # Define the operator sum
    os = OpSum()
    for b in 1:(L - 1)
        os += -mu - U*0.5, "N", b
        os += U*0.5, "N", b, "N", b
        os += -J, "A", b, "Adag", b + 1
        os += -J, "Adag", b, "A", b + 1
    end
    os += -mu - U*0.5, "N", L
    os += U*0.5, "N", L, "N", L
    if bc == 1
        os += -J, "A", 1, "Adag", L
        os += -J, "Adag", 1, "A", L
    end

    # Construct the Hamiltonian as an MPO
    H = MPO(os, sites)

    # Initialize a random MPS
    psi0 = randomMPS(sites)
    
    # Run DMRG to find the ground state
    energy, psi = dmrg(H, psi0; nsweeps, maxdim, cutoff)

    # Calculate the correlation matrix
    adagacorr = correlation_matrix(psi, "Adag", "A")
    npzwrite(file_name, adagacorr)

    return adagacorr, psi, energy
end
\end{lstlisting}
\end{widetext}
