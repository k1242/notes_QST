
\appendix


\begin{widetext}
\section{Code listing} \label{app:codes}
Please copy your code in the appendix.
\begin{lstlisting}[language=Python]
"""

Module to generate the Hamiltonian of the transverse field Ising model.

H = -J sum_i sigma^x_i sigma^x_{i+1} - g sum_i sigma^z i.

Used in the solution of exercise 5.1

"""

import numpy as np
import scipy
from scipy import sparse
import scipy.sparse.linalg
import matplotlib.pyplot as plt

Id = sparse.csr_matrix(np.eye(2))
Sx = sparse.csr_matrix([[0., 1.], [1., 0.]])
Sz = sparse.csr_matrix([[1., 0.], [0., -1.]])
Splus = sparse.csr_matrix([[0., 1.], [0., 0.]])
Sminus = sparse.csr_matrix([[0., 0.], [1., 0.]])


def singesite_to_full(op, i, L):
    op_list = [Id]*L  # = [Id, Id, Id ...] with L entries
    op_list[i] = op
    full = op_list[0]
    for op_i in op_list[1:]:
        full = sparse.kron(full, op_i, format="csr")
    return full


def gen_sx_list(L):
    return [singesite_to_full(Sx, i, L) for i in range(L)]


def gen_sz_list(L):
    return [singesite_to_full(Sz, i, L) for i in range(L)]


def gen_hamiltonian_periodic(sx_list, sz_list, g, J=1.):
    """ assumes periodic boundery conditions """
    L = len(sx_list)
    H = sparse.csr_matrix((2**L, 2**L))
    for j in range(L):
        H = H - J *( sx_list[j] * sx_list[(j+1)%L])
        H = H - g * sz_list[j]
    return H
\end{lstlisting}
\end{widetext}
