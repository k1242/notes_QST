%  shares impressions left by one of the books
%  analyzes choices the writer made that led to those impressions


% Has the student applied his/her writing to the task? (Detailed, evidence-based analysis of how the author achieves certain effects.)
% Does the paper include a thesis statement that includes an opinion or unique “angle” on the book?
% Is the paper logically organized?

% Do paragraphs begin with strong topic sentences?

% Are sentences reader-friendly (not too long or convoluted)?

% Is spelling, capitalization, paragraphing, and typography correct?

% Are verbs conjugated correctly?

% Has the student included proper citations where appropriate?

% Has the student written about 1,000 words? Is the student’s name on the paper? Is the assignment in an editable format (Word)?

\phantom{42} \vspace{-2cm}

\subsection*{Obsession, Isolation, and the Possibility of Change in <<I’m a Fan>>}

Sheena Patel’s <<I’m a Fan>> leaves a powerful and complex emotional impression. It is a story that dives deep into the psyche of a protagonist trapped in cycles of obsession, inner turmoil, and societal disconnection. Reading the novel is often a heavy, suffocating experience, as Patel captures the darker aspects of human dependency and self-doubt. Yet, amidst the tension, the book gestures toward the potential for personal growth. Through its fragmented narration, sharp material imagery, and stark contrasts between despair and transformation, <<I’m a Fan>> creates an intense, layered experience that lingers long after the final page.

An even greater sense of intimacy in the narrative, the feeling that I was living this story myself, came from the almost complete absence of names. This also contributed to a sense of depersonalization, reducing individuals to their roles or functions. It wasn’t a person but "the woman she is obsessed with." It wasn’t a man but "the man she wants to be with." For me, this also highlighted the one-sided nature of the perspective available to us. We don’t just learn the story in fragments, scattered across different moments in time and space—we experience it exclusively from the protagonist’s point of view. Perhaps the events themselves were different, but in terms of the evolution of this specific individual (the protagonist), that difference doesn’t matter.

One of the most striking aspects of <<I’m a Fan>> is its fragmented, diary-like structure. This style fosters an intimacy that draws the reader deep into the protagonist’s mind while also creating a suffocating sense of entrapment. The rhythmic urgency in how the protagonist recounts her obsessive routines—such as refreshing Instagram or analyzing others’ lives—intensifies the experience. The immediacy of her words pulls the reader into her spiraling thoughts in a way that feels both engrossing and unsettling. This narrative style evokes a similar sense of intimacy that I found in Turgenev’s <<The Diary of a Superfluous Man.>> Like Turgenev’s protagonist, Patel’s narrator invites the reader to witness her inner world without filters. However, while Turgenev’s work emphasizes societal alienation, Patel shifts the focus to internal conflict, making the sense of isolation more suffocating and personal. Both works explore the passivity of their protagonists, which at times began to weigh on me, evoking a sense of unease that shaped my experience of the book.

The novel also uses material details to reflect the protagonist’s inner struggles. Her descriptions of wealth and aesthetics highlight both her admiration for and alienation from the lives she observes. For example, her fascination with bespoke doors in affluent neighborhoods reflects what I interpreted as a longing to belong to a world she feels excluded from. At the same time, it underscores the futility of relying on material symbols for validation. These material details resonated with me on a psychological level, even if their significance didn’t immediately register. They suggest how the protagonist, like many in a hyper-visual modern culture, becomes trapped by her own perceptions of what she should aspire to. While these descriptions didn’t evoke a sense of being trapped for me personally, they contributed to the novel’s oppressive atmosphere. This tension between external appearances and internal emptiness reminded me of Dostoevsky’s explorations of the human condition in <<Crime and Punishment.>> However, Patel’s use of material imagery feels more grounded in contemporary culture, emphasizing emotional rather than spiritual conflict. Like Dostoevsky’s Raskolnikov, who is ultimately redeemed through suffering and faith, Patel’s protagonist is also in an emotional abyss. Yet the steps she takes at the end of the book toward setting boundaries and learning self-acceptance felt deeply significant to me. Though her hesitation about meeting the man she’s obsessed with seems minor externally, it struck me as a meaningful and almost optimistic step forward.

Patel’s exploration of relationships is another area where the novel feels particularly suffocating. The protagonist’s connection with the man she desires is marked by imbalance and dependency. Rather than offering her a sense of safety or belonging, the relationship amplifies her feelings of inadequacy. The protagonist seems painfully aware of her own dispensability, recognizing that she is treated as a secondary figure in the lives of others. This portrayal of relationships as battlegrounds rather than sources of connection highlights the protagonist’s internal struggles. I found myself genuinely intrigued by this dynamic. It felt like a warped mirror, reflecting unhealthy patterns in relationships in a way that forced me to confront their discomfort head-on.

Despite the heaviness of the novel, <<I’m a Fan>> leaves room for small but significant moments of change. The protagonist begins to question her own behaviors and consider what she wants for herself, rather than solely centering her life around the expectations or validation of others. This shift is subtle but important, suggesting that even in a world dominated by superficial measures of success and love, self-awareness can be a first step toward personal growth. Unlike Turgenev’s superfluous man, who succumbs to despair, or Dostoevsky’s Raskolnikov, whose redemption comes through spiritual suffering, Patel’s protagonist finds her way toward self-acceptance through quieter means. Her progress is incomplete, but it feels genuine, reflecting the complexities of navigating modern life.

Reading <<I’m a Fan>> was a challenging yet rewarding experience. Sheena Patel uses fragmented narration, material symbolism, and emotionally charged contrasts to create a novel that feels both suffocating and deeply reflective. The book captures the weight of obsession and isolation while leaving space for the possibility of transformation. Though heavy and tense in tone, the novel’s conclusion suggests that change is possible, even if it begins in small steps. <<I’m a Fan>> is not a story of grand redemption but of incremental shifts—offering a poignant reminder of the resilience required to face modern struggles with self-perception and relationships.


\phantom{42} \hfill \textit{\textbf{Khoruzhii Kirill}}