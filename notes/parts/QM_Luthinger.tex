\subsection*{A brief intro into Luthinger liquids}

% fermi-liq -- remarkable
% бозонизация

% Field Theories of Condensed Matter Physics, Книга, Эдуардо Фрадкин
% Bruus Flensberg

Let's look at the density correlation functions
\begin{equation*}
	D(\vc{q}, \omega) = \int \frac{d^D p}{d (2\pi)^D}
	\frac{n_p - n_{p+q}}{\omega - \varepsilon(p+q) + \varepsilon(p) + i \varepsilon} = 
	\frac{q}{2\pi} \left(
		\frac{1}{\omega - q v_F + i \varepsilon} - \frac{1}{\omega + q v_F + i \varepsilon}
	\right).
\end{equation*}
The spinless Tomonaga0Luthinger model
\begin{equation*}
	H = H_0 + H_{int} = \sum_k \varepsilon_k c\D_k c_k + \frac{1}{2L} \sum_{k,k',q} V(q) c_k\D c_{k'}\D c_{k'-q} c_{k+q}.
\end{equation*}
We could work with the linear dispersion for low energy
\begin{equation*}
	\xi_k = \varepsilon_k - \mu \approx (|k|-k_F) v_F.
\end{equation*}
Next we separate rightmovers and leftmovers
\begin{equation*}
	c_k = c_{kR} \theta(k) + c_{kL} \theta(-k),
\end{equation*}
and than
\begin{equation*}
	H_0 = v_F \sum_{k > 0} \left(
		k c_{kR}\D c_{kR} - k c_{kL}\D c_{kL}
	\right) - \left(N_R+N_L\right) k_F v_F.
\end{equation*}
There are two interactions
\begin{equation*}
	H_{int}^{(1)} = \frac{1}{2L} \sum_{k > 0, q, k' < 0} V(q) \left(
		c_{kR}\D c_{k'L}\D c_{k'-q,L} c_{k+q,R} + c_{kR}\D c_{k'L}\D c_{k'-q,R} c_{k+q,L}
	\right) + (R \leftrightarrow L).
\end{equation*}
Note that $k \approx k_F$, $q \approx \pm-2k_F$. 
Than we could rewrite it in form
\begin{equation*}
	H_{int}^{(1)} = \frac{1}{2L} \sum \left(
		V(q) c_{kR}|\D c_{k+q,R} c\D_{k' L} c_{k'-q,L} - V(-q + k'-k) c\D_{kR} c_{k+q,R} c\D_{k'c} c_{k'-q,L}
	\right).
\end{equation*}
Then we can introduce left/right moving density operators
\begin{equation*}
	\rho_R(q) = \sum_{k>0} c_k\D c_{k+q} \approx \sum_{k>0} c\D_{kR} c_{k+qR},
\end{equation*}
and rewrite
\begin{equation*}
	H_{int}^{(1)} \approx \frac{1}{2L} \sum_q \left(
		V(0) - V(2 k_F)
	\right) \rho_R(q) \rho_L(-q) + (R \leftrightarrow L),
\end{equation*}
and finally we have
\begin{equation*}
	H_{int} = \frac{1}{2L} \sum_{q \neq 0} V_1 \left(
		\rho_L (q) + \rho_R(q)
	\right) \left(
		\rho_L(-q) + \rho_R(-q)
	\right).
\end{equation*}

% flat band systems



\subsection*{Real space representation}

We work with
\begin{equation*}
	\rho_{R,L}(x) = \frac{1}{L} \sum_q e^{i q x} \rho_{R,L} (q) + \rho_{0, R,L},
\end{equation*}
and still
\begin{equation*}
	\left[\rho_R(x), \rho_R(x')\right] = \frac{1}{L^2} \sum_{q,q'} e^{i q x - i q' x'} \left[
		\rho_R(q), \rho_R(-q')
	\right] = \frac{1}{2\pi L} \sum_q q e^{ i q(x-x')} = \frac{1}{2\pi i} \partial_x \delta(x-x'),
\end{equation*}
for left 
\begin{equation*}
	[\rho_L(x), \rho_L(x')] = \ldots = - \frac{1}{2\pi i} \partial_x \delta(x-x').
\end{equation*}
These relations are called the Kac-Moody algebra.  The Hamiltonian could be rewritten as 
\begin{equation*}
	H = \int_{-L/2}^{L/2} dx\ \left(
		\pi v_F \left(
			\rho_R(x) \rho_R(x) + \rho_L(x) \rho_L(x)
		\right) + C(N_L,N_R) + \frac{V_1}{2}\left(
			\rho_R(x) + \rho_L(x)
		\right) \left(
			\rho_R(x) + \rho_L(x)
		\right)
	\right).
\end{equation*}
Now we wnat to define
\begin{align*}
	\tfrac{1}{\sqrt{\pi}} \partial_x \varphi(x) &= \rho_R(x) + \rho_L(x) - \rho_{R_0} - \rho_{L_0},  \\
	- \tfrac{1}{\sqrt{\pi}} P(x) &= (\rho_R(x) - \rho_L(x) - (\rho_{R_0} - \rho_{L_0}).
\end{align*}
We see that $\varphi(x)$ and $P(x)$ are conjugate fields
\begin{equation*}
	\left[\varphi(x), P(x)\right] = i \delta(x-x').
\end{equation*}
Then we can rewrite the Hamiltonian
\begin{equation*}
	H = \frac{\tilde{V}}{2} \int_{-L/2}^{L/2} dx\ \left(
		g P(x)^2 + \frac{1}{g} (\partial_x \varphi(x))^2
	\right),
	\hspace{10 mm} 
	\tilde{V} = \frac{1}{g} v_F,
	\hspace{5 mm} 
	g = \sqrt{1 + \frac{V_1}{v_F}}.
\end{equation*}
Thus we can calculate
\begin{equation*}
	\langle \rho(x) \rho(x')\rangle \sim \left(\frac{1}{x}\right)^{2g} \cos(2 \pi \rho_0 x).
\end{equation*}



\subsection*{Majumdar–Ghosh model}


Consider frustrating Hamiltonian
\begin{equation*}
	H = J \sum_j \vc{S}_j \cdot \vc{S}_{j+1} + J' \sum_j \vc{S}_j \cdot \vc{S}_{j+2}.
\end{equation*}
For the special point $J' = J/2$ the g.s. can be obtained exactly. We can rewrite the Hamiltonian
\begin{equation*}
	H_{MG} = \frac{J}{2} \sum_j \left(
		\vc{S}_j \cdot \vc{S}_{j+1} + \vc{S}_{j} \cdot \vc{S}_{j-1} + \vc{S}_{j-1} \cdot \vc{S}_{j+1}
	\right) = \frac{J}{4} \sum_j \left(
		\vc{S}_{j-1} + \vc{S}_j + \vc{S}_{j+1}
	\right)^2 + \const
\end{equation*}
The exact ground state is
\begin{equation*}
	\ket{d_{\pm}} = \prod_{n=1}^{N/2} \left(
		\ket{\uparrow_{2n}} \ket{\downarrow_{2n \pm 1} } - \ket{\downarrow_{2n}} \ket{\uparrow_{2n \pm 1}}
	\right).
\end{equation*}
The main idea is to show that $H_{MG}$ is a sum of projectors.

To proof that $\ket{d_{\pm}}$ is ground state we need to consider
\begin{equation*}
	\vc{J}_i = \vc{S}_{i-1} + \vc{S}_i + \vc{S}_{i+1},
\end{equation*}
with $J^2 = J (3+1)$ and $J = \frac{1}{2}, \frac{3}{2}$. And we can rewrite $H_{MG}$ in terms of the projectors
\begin{equation*}
	H_{MG} = \frac{JJ}{8} \sum_j \hat{P}_{3/2} (j-1,j,j+1),
	\hspace{10 mm} 
	\hat{P}_{3/2}(j-1,j,j+1) = \frac{1}{3}\left(J^2 - \frac{3}{4}\right).
\end{equation*}

The spin correlations is
\begin{equation*}
	\langle \vc{S}_i \cdot \vc{S}_j\rangle = \left\{\begin{aligned}
	    &3/4, &i=j \\
	    &-3/4, &i,j \in \text{bond} \\
	    &0, \text{otherwise}
	\end{aligned}\right.
\end{equation*}
VBS -- nonzero correlations singlet bonds. 




\subsection*{Violate LSM}

Consider example that violate LSM:
\begin{equation*}
	H_D = \sum_{i \in \text{even}} \left(J_1 \vc{S}_i \cdot \vc{S}_{i+1} + J_2 \vc{S}_i \vc{S}_{i-1}\right).
\end{equation*}
The gap remains open for all $J_2 < J_1$. For $J_2 \to - \infty$ the pairs of spins on the $J_2$ bonds are forced to be parallel (in triplets), so we go to the $S=1$ Hamiltonian.



\subsection*{Haldane gap: AKLT model}


AKLT (Affleck-Kennedy-Lieb-Tasaki model) Hamiltonian is
\begin{equation*}
	H = \sum_j \left(
		\vc{S}_j \cdot \vc{S}_{j+1} + \alpha \left(\vc{S}_j \cdot \vc{S}_{j+1}\right)^2
	\right),
\end{equation*}
with $\alpha=1/3$ the model has an exact ground state. 

The spin-two projector
\begin{equation*}
	P_2 (j, j+1) = \frac{1}{2} \vc{S}_j \vc{S}_{j+1} + \frac{1}{6}\left(\vc{S}_j \vc{S}_{j+1}\right) + \frac{1}{3} = S^2_{j,j+1} \left(
		S^2_{j,j+1}-2
	\right),
\end{equation*}
this annihilates states with spin 0 or 1.
The Hamiltonian	is nothing but
\begin{equation*}
	H = J \sum_j \left( 2 P_2 (j, j+1) - \frac{2}{3}\right).
\end{equation*}
% фермионы абрикосова
The key idea of AKLT was to introduce two auxilaries spin $\frac{1}{2}$ on each site. 

Our projector is
\begin{equation*}
	P = \kb{1}{\uparrow \uparrow}  + \frac{1}{\sqrt{2}}\ket{0}\left(\bra{\uparrow \downarrow} + \bra{\downarrow \uparrow}\right) + \kb{-1}{\downarrow \downarrow}. 
\end{equation*}
The ground state is
\begin{equation*}
	\ket{gs} = \left(\prod_j \hat{P}_j\right) \prod_{i=1}^N \left(
		\ket{\uparrow}_{i,R} \ket{\downarrow}_{i+1,L} - \ket{\downarrow}_{i,R} \ket{\uparrow}_{i+1,L}
	\right).
\end{equation*}
In $\hat{S}^z$ measurements we have hidden order -- non-local string order parameter with LRO
\begin{equation*}
	\langle S^z_j \exp\left(i \pi \sum_{k=j+1}^{j+r-1} S_k^z\right) S^z_{j+r}\rangle,
\end{equation*}
that is finite in the entire Haldane phase.

% \begin{equation*}
% 	\Gamma(r) = \langle b\D_r b_0\rangle \propto r^{-K/2}
% \end{equation*}

% $K = K_c = 1/2$ -- phase transition

