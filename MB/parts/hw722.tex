
\textbf{2. Bogoliubov transformation}. 
$\hat{H}$ can be diagonalized with a Bogoliubov transformation to a new set of creation and annihilation operators
\begin{equation}
	\left.\begin{aligned}
    	\hat{a}_p\D &= u_p \hat{\alpha}_p\D + v_p \hat{\alpha}_{-p}, \\
		\hat{a}_p &= u_p \hat{\alpha}_p + v_p \alpha_{-p}\D.
	\end{aligned}\right.
	\label{Btransform}
\end{equation}
The newly introduced $\hat{\alpha}_p$  and $\hat{\alpha}_p$ have to obey bosonic commutation relations (canonical transformation):
\begin{equation*}
	[\hat{a}_p, \hat{a}_p\D] = 1 = u_p^2 (\hat{\alpha}_p \hat{\alpha}_p\D-\hat{\alpha}_p\D \hat{\alpha}_p) + v_p^2 \left(\hat{\alpha}_{-p}\D \hat{\alpha}_{-p}-\hat{\alpha}_{-p} \hat{\alpha}_{-p}\D\right) + u_p v_p \left(
		\hat{\alpha}_p \hat{\alpha}_{-p} - \hat{\alpha}_{-p} \hat{\alpha}_p + \hat{\alpha}_{-p}\D \hat{\alpha}_{p}\D + \hat{\alpha}_{p}\D \hat{\alpha}_{-p}\D
	\right) = u_p^2 - v_p^2.
\end{equation*}
That allows for a convenient parametrization of the form $u_p = \ch \theta_p$, $v_p = \sh \theta_p$ with\footnote{
	We can do this because there are no external fields imposed on the system.
} $u,v \in \mathbb{R}$. In principle this is the same as substitution of the form $u_p = (1-A_p^2)^{-1/2}$ and $v_P = A_p (1-A_p^2)^{-1/2}$.


