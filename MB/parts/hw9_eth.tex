
\textbf{Thermalization}. Let's start by building some intuition about what could be called thermalization for an isolated quantum system \cite{khlebnikov_thermalization_2014}. Let the initial state be given by $\ket{\psi_0}$, then, in the basis of energy eigenstates $\ket{j}$, the evolution 
\begin{equation*}
	\ket{\psi(t)} = \sum_{j=1}^{\mathcal{N}} c_j e^{- i \varepsilon_j t} \ket{E_j}
\end{equation*}
with $c_j = \bk{E_j}{\psi_0}$, $\varepsilon_j = \bk{j}[\hat{H}]{j}$ and $\mathcal{N} = \dim H$. For some observable $\hat{A}(t)$ the mean value could be expressed as 
\begin{equation}
	A(t) = \bk{\psi(t)}[\hat{A}]{\psi(t)} = \sum_{j,k} \bar{c}_k c_j e^{-i (\varepsilon_j - \varepsilon_k)} \bk{k}[\hat{A}]{j}
	=
	\sum_j |c_j|^2 \bk{j}[\hat{A}]{j} + \sum_{k \neq j} c_j \bar{c}_k e^{-i (\varepsilon_j - \varepsilon_k)} \bk{k}[\hat{A}]{j}.
	\label{eq:At}
\end{equation}
After some time of thermalization $\tth$ we would like to see that the observables reach thermal values (independent of the initial conditions) with small fluctuations around (fig. \ref{fig:BASE}, \red{red curve})
\begin{equation*}
	A(t \gg \tth) = A(E) + \text{small fluctuations},
	\hspace{10 mm} 
	E = \bk{\psi_0}[\hat{H}]{\psi_0}.
\end{equation*}
To achieve small fluctuations around the average value, as we see from \eqref{eq:At}, it is enough to require the smallness of the off-diagonal elements\footnote{
	Note that the number of diagonal terms is $\mathcal{N}$ and off-diagonal $\mathcal{N}^2- \mathcal{N}$. If we consider the contribution of each off-diagonal term to be random, then the fluctuations can be estimated as $\sqrt{\mathcal{N}^2} |\bk{k}[\hat{A}]{j}|$, which leads to to the requirement of smallness.
}. 
% что и составляет первое условие Eigenstate Thermalization Hypothesis (ETH)
And so that $A(E)$ does not depend on the initial conditions, we can consider the case when diagonal elements are smooth functions of energy
\begin{equation*}
	\bk{j}[\hat{A}]{j} = A(\varepsilon_j).
\end{equation*}
Indeed, then for the initial state lying in $\Delta E $ such that the spread $\partial_E A(E) \Delta E$ is small, the final result is
\begin{equation*}
	A(t \gg \tth) \approx \sum_j |c_j|^2 \bk{j}[\hat{A}]{j} \approx  A(E).
\end{equation*}
This is how we come to the formulation of the Eigenstate Thermalization Hypothesis (ETH), put forward by Deutsch \cite{PhysRevA.43.2046} and Srednicki \cite{PhysRevE.50.888}: if off-diagonal terms $\bk{k}[\hat{A} ]{j}$ are small in compared to diagonal and diagonal terms are smooth functions of energty, then the observed one seems to be thermalized. It is worth making some reservation that for an isolated system a pure state remains pure $\tr \rho^2 = 1$, while for a thermal state $\tr \rho^2 < 1$, which is why we talk about the thermalization of observables\footnote{
	The hope arises that if we divide the system into two subsystems $\Omega_1 \cup \Omega_2$, then $\rho_1 = \tr_{\Omega_2} \rho$ can actually turn out to be thermal, and $\Omega_2$ acts in some sense thermostat for $\Omega_1$. This assumption will not be developed within the framework of this essay, but the I hope to return to this issue later.
}. Again, judging by the conditions, it seems that systems and observables $A_1,\, A_2$ are possible such that $A_1$ is thermalized, but $A_2$ is not.

The main conclusion of this section: sometimes it happens that in some system and for some observable $\hat{A}$ its value $A(t \gg \tth)$ reaches a constant with small fluctuations around, most likely this will correspond to the indicated ETH terms.