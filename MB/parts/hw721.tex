
\textbf{1. Hamiltonian}. 
Consider a microscopic Hamiltonian for bosons with weak contact interactions:
\begin{equation}
	\hat{H} - \mu \hat{N} = \sum_p (\varepsilon_p - \mu) \hat{a}_p\D \hat{a}_p + \frac{\varphi}{2V} \sum_{p, p', q} \hat{a}\D_{p+q} \hat{a}\D_{p'-q} \hat{a}_{p'} \hat{a}_p,
	\label{BECbase}
\end{equation}
where $\varepsilon_p = p^2 / 2m$ and second term as $\hat{V}$.  For $u = 0$ the groundstate in a grandcanonical description is a coherent
state of bosons in the zero-momentum state, i.e. all particles are Bose condensed. 

Finite interactions lead to scattering of bosons from the condensate into finite momentum modes
and hence a depletion of the condensate fraction. However, if the interactions are weak,
one can still assume that the $p = 0$ mode is macroscopically occupied, $\langle a_0\D a_0\rangle \gg 1$. As $[a_0, a_0\D]=1$, one can neglect it for a macroscopically occupied $p = 0$ mode and replace $a_0,\ a_0\D$ by their expectation value $\sqrt{N_0}$, the number of bosons in the condensate. Thus our small parameter is $(N-N_0)/N_0$. One can therefore approximate all other modes to be small $a_p \ll \sqrt{N_0}$ and  therefore neglect all terms in the interaction part of above Hamiltonian which contain more than two creation/annihilation operators with $p \neq 0$. 


\begin{figure}[h]
    \centering
% a
\begin{tikzpicture}
	\begin{feynman}[small]
	\vertex (i1);
	\vertex [below right=of i1] (a);
	\vertex [below left=of a] (i2);
	\vertex [right=of a] (b);
	\vertex [above right=of b] (f1);
	\vertex [below right=of b] (f2);
	\diagram* {
	(i1) -- [scalar] (a) -- [scalar] (i2),
	(f1) -- [scalar] (b) -- [scalar] (f2),
	(a) -- [boson] (b),
	};
	\end{feynman}
\end{tikzpicture}
    \caption{Interaction of condensate particles}
    \label{fig:BCa}
\end{figure}


The leading term of the expansion involves interactions solely between the stationary particles (particles of the condensate) as in fig. \ref{fig:BCa} (the dashed line corresponds to condensed particles)
\begin{equation*}
	\hat{V}_0 = \frac{\varphi}{2V} \hat{a}\D_0 \hat{a}\D_0 \hat{a}_0 \hat{a}_0.
\end{equation*}
There are no terms that contain only one creation or annihilation operator for non-condensate particles due to the conservation of momentum.


\begin{figure}[h]
    \centering
% б
\begin{tikzpicture}
	\begin{feynman}[small]
	\vertex (i1);
	\vertex [below right=of i1] (a);
	\vertex [below left=of a] (i2);
	\vertex [right=of a] (b);
	\vertex [above right=of b] (f1);
	\vertex [below right=of b] (f2);
	\diagram* {
	(i1) -- [scalar] (a) -- [fermion, edge label=\(p\)] (i2),
	(f1) -- [fermion, edge label'=\(p\)] (b) -- [scalar] (f2),
	(a) -- [boson] (b),
	};
	\end{feynman}
\end{tikzpicture}
\hspace{5 mm} 
% в
\begin{tikzpicture}
	\begin{feynman}[small]
	\vertex (i1);
	\vertex [below right=of i1] (a);
	\vertex [below left=of a] (i2);
	\vertex [right=of a] (b);
	\vertex [above right=of b] (f1);
	\vertex [below right=of b] (f2);
	\diagram* {
	(f1) -- [scalar] (b) -- [fermion, edge label'=\(p\)] (f2),
	(i1) -- [fermion, edge label=\(p\)] (a) -- [scalar] (i2),
	(a) -- [boson] (b),
	};
	\end{feynman}
\end{tikzpicture}
\hspace{5 mm} 
% г
\begin{tikzpicture}
	\begin{feynman}[small]
	\vertex (i1);
	\vertex [below right=of i1] (a);
	\vertex [below left=of a] (i2);
	\vertex [right=of a] (b);
	\vertex [above right=of b] (f1);
	\vertex [below right=of b] (f2);
	\diagram* {
	(i1) -- [fermion, edge label=\(p\)] (a) -- [fermion, edge label=\(p\)] (i2),
	(f1) -- [scalar] (b) -- [scalar] (f2),
	(a) -- [boson] (b),
	};
	\end{feynman}
\end{tikzpicture}
\hspace{5 mm} 
% г
\begin{tikzpicture}
	\begin{feynman}[small]
	\vertex (i1);
	\vertex [below right=of i1] (a);
	\vertex [below left=of a] (i2);
	\vertex [right=of a] (b);
	\vertex [above right=of b] (f1);
	\vertex [below right=of b] (f2);
	\diagram* {
	(f1) -- [fermion, edge label'=\(p\)] (b) -- [fermion, edge label'=\(p\)] (f2),
	(i1) -- [scalar] (a) -- [scalar] (i2),
	(a) -- [boson] (b),
	};
	\end{feynman}
\end{tikzpicture}
    \caption{Interaction of condensate particle and  non-condensate particle}
    \label{fig:BCb}
\end{figure}


The next four expansion terms each contain one operator of creation and one operator of annihilation above the non-condensate particles (fig. \ref{fig:BCb}):
\begin{equation*}
	\hat{V}_2 = \frac{\varphi}{2V}  \sum_{p \neq 0} \left(
		\hat{a}\D \hat{a}_0 \hat{a}_0 \hat{a}_p + 
		\hat{a}_p\D \hat{a}_0\D \hat{a}_p \hat{a}_0 + 
		\hat{a}_0 \hat{a}\D \hat{a}_0 \hat{a}_p + 
		\hat{a}_p\D \hat{a}_0\D \hat{a}_p \hat{a}_0
	\right) \overset{1}{=}  \frac{2\varphi}{V} \hat{a}\D_0 \hat{a}_0 \sum_{p \neq 0} \hat{a}_p\D \hat{a}_p,
\end{equation*}
where in $ \overset{1}{=} $ it was used that $[\hat{a}_0, \hat{a}_p] = 0$. 




% https://arxiv.org/ftp/arxiv/papers/1601/1601.05437.pdf
% https://ftp.fau.de/ctan/graphics/pgf/base/doc/pgfmanual.pdf

\begin{figure}[h]
    \centering
\begin{tikzpicture}
	\begin{feynman}[small]
	\vertex (i1);
	\vertex [below right=of i1] (a);
	\vertex [below left=of a] (i2);
	\vertex [right=of a] (b);
	\vertex [above right=of b] (f1);
	\vertex [below right=of b] (f2);
	\diagram* {
	(i1) -- [fermion, edge label=\(p\)] (a) -- [scalar] (i2),
	(f1) -- [fermion, edge label'=\(-p\)] (b) -- [scalar] (f2),
	(a) -- [boson] (b),
	};
	\end{feynman}
\end{tikzpicture}
\hspace{5 mm} 
\begin{tikzpicture}
	\begin{feynman}[small]
	\vertex (i1);
	\vertex [below right=of i1] (a);
	\vertex [below left=of a] (i2);
	\vertex [right=of a] (b);
	\vertex [above right=of b] (f1);
	\vertex [below right=of b] (f2);
	\diagram* {
	(i1) -- [scalar] (a) -- [fermion, edge label=\(p\)] (i2),
	(f1) -- [scalar] (b) -- [fermion, edge label'=\(-p\)] (f2),
	(a) -- [boson] (b),
	};
	\end{feynman}
\end{tikzpicture}
    \caption{Creation and annihilation of two condensate particles from non-condensate}
    \label{fig:BCc}
\end{figure}


Two more terms of the expansion contain two creation operators each
and two operators for annihilation condensate particles (fig. \ref{fig:BCc}):
\begin{equation*}
	\hat{V}_2' = \frac{\varphi}{2V} \sum_{p \neq 0} \left(
		\hat{a}\D_0 \hat{a}\D_0 \hat{a}_p \hat{a}_{-p} + \hat{a}\D_p \hat{a}\D_{-p} \hat{a}_0  \hat{a}_0
	\right)
\end{equation*}
Let's substitute into the equations $\hat{a}_0 = \hat{a}_0\D = \sqrt{N_0}$ and rewrite it in terms $N = N_0 + \sum_{p \neq 0} \hat{a}_p\D \hat{a}_p$. The quadratic terms in the number of non-condensate particles should be discarded. Thus, the complete Hamiltonian can be represented in the following form:
\begin{equation}
	\hat{H} = \frac{N^2}{2V}\varphi  + \sum_{p \neq 0} \epsilon_p \hat{a}_p\D \hat{a}_p + \frac{N}{2V} \sum_{p \neq 0} \left(\hat{a}\D_p \hat{a}\D_{-p} + \hat{a}_p \hat{a}_{-p}\right),
	\hspace{10 mm} 
	\epsilon_p = \varepsilon_p + \varphi n.
	\label{H1}
\end{equation}
We can assume that the total number of particles is fixed and the number of condensate particles is variable, then we can work with the Hamiltonian in form \eqref{H1}. 

