

\textbf{Wick's theorem}. Note that from \eqref{gen} are convenient to obtain Wick’s theorem \red{maybe}. Expanding \eqref{gen} in the Taylor series we have from the LHS
\begin{equation*}
	\langle e^{\sum_j b_j x_j}\rangle = 1 + \frac{1}{2!} \sum_{i,j} b_i b_j \langle x_i x_j\rangle + \frac{1}{4!}  \sum_{i,j,k,l} b_i b_j b_k b_l \langle x_i x_j x_k x_l\rangle + \ldots
\end{equation*}
and from the RHS 
\begin{equation*}
	e^{\frac{1}{2}\sum_{i,j} b_i \langle x_i x_j \rangle b_j} 1 + \frac{1}{2} \sum_{i,j} b_i b_j \langle x_i x_j\rangle + \sum_{i,j,k,l} b_i b_j b_k b_l \langle x_i x_j\rangle \langle x_k x_l\rangle + \ldots,
\end{equation*}
so collecting terms with proper $B^4$ we get
\begin{equation*}
	\langle x_i x_j x_k x_l\rangle = \langle x_i x_j\rangle \langle x_k x_l\rangle + \langle x_i x_k\rangle \langle x_j x_l\rangle + \langle x_i x_l\rangle \langle x_j x_k\rangle.
\end{equation*}
This result is known as Wick's theorem.