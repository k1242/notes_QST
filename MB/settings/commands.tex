% базовая подстройка
\renewcommand{\d}{\, d}
\renewcommand{\leq}{\leqslant}
\renewcommand{\geq}{\geqslant}

% \textbf
\newcommand{\addletter}[2]{\begin{picture}(7,7) \put(0,#1){{#2)}} \end{picture}}


% tmp
\newcommand{\tauint}{\int_{0}^{\beta} \hspace{-1mm} d \tau\ }
\newcommand{\tn}[1]{(\textbf{#1})}
\newcommand{\nBF}{n_\text{\scalebox{0.7}{BF}}}
\newcommand{\Hc}{\sub{h}{c}}
\newcommand{\Z}{\mathcal{Z}}
\newcommand{\Kc}{\sub{K}{c}}
\newcommand{\gs}{\ket{\textnormal{gs}}}
\newcommand{\F}{\mathcal{D}}
\renewcommand{\O}{\mathcal{O}}

% авторские команды
\newcommand{\te}[1]{\text{#1}}
\newcommand{\vc}[1]{\boldsymbol{#1}}
\newcommand{\1}{\mathbbm{1}}
\newcommand{\T}{^{\textnormal{T}}}
\newcommand{\D}{^{\dag}}
\newcommand{\sub}[2]{#1_{\textnormal{#2}}}
\newcommand{\vp}{\vphantom{\dfrac{1}{2}}}
\newcommand{\hc}{\textnormal{h.c.}}

% операторы (просто прямой текст)
\renewcommand{\Im}{\mathop{\mathrm{Im}}\nolimits}
\renewcommand{\Re}{\mathop{\mathrm{Re}}\nolimits}
% \renewcommand{\P}{\mathop{\mathrm{P}}\nolimits}
% \newcommand{\E}{\mathop{\mathrm{E}}\nolimits}
% \newcommand{\D}{\mathop{\mathrm{D}}\nolimits}
% \newcommand{\cov}{\mathop{\mathrm{cov}}\nolimits}
\newcommand{\diag}{\mathop{\mathrm{diag}}\nolimits}
\newcommand{\card}{\mathop{\mathrm{card}}\nolimits}
\newcommand{\grad}{\mathop{\mathrm{grad}}\nolimits}
\renewcommand{\div}{\mathop{\mathrm{div}}\nolimits}
\newcommand{\rot}{\mathop{\mathrm{rot}}\nolimits}
\newcommand{\Ker}{\mathop{\mathrm{ker}}\nolimits}
\newcommand{\spec}{\mathop{\mathrm{spec}}\nolimits}
\newcommand{\sign}{\mathop{\mathrm{sign}}\nolimits}
\newcommand{\tr}{\mathop{\mathrm{tr}}\nolimits}
\newcommand{\rg}{\mathop{\mathrm{rg}}\nolimits}
\newcommand{\const}{\textnormal{const}}


\renewcommand{\th}{\mathop{\mathrm{tanh}}\nolimits}
\newcommand{\cth}{\mathop{\mathrm{coth}}\nolimits}
\newcommand{\sh}{\mathop{\mathrm{sinh}}\nolimits}
\newcommand{\ch}{\mathop{\mathrm{cosh}}\nolimits}

% цветной текст
\newcommand{\red}[1]{\textcolor{red}{#1}}
\newcommand{\green}[1]{\textcolor{urlcolor}{#1}}
\newcommand{\blue}[1]{\textcolor{ublue}{#1}}


% символы
\newcommand{\cmark}{\text{\ding{51}}}
\newcommand{\xmark}{\text{\ding{55}}}


% подгрузка pdf_tex картинок
% \newcommand{\incfig}[1]{%
%     \def\svgwidth{\columnwidth}
%     \import{./figures/}{#1.pdf_tex}
% }


% специфично к квантам
\newcommand{\ket}[1]{\left| #1 \right\rangle}
\newcommand{\bra}[1]{\left\langle #1 \right|}

% \newcommand{\dppp}{\frac{d^3 p}{(2 \pi \hbar)^3}}

\DeclareDocumentCommand{\bk}{m o m}{
    \IfNoValueTF{#2}{\langle #1 | #3 \rangle}{\langle #1 | #2 | #3 \rangle}
}

\DeclareDocumentCommand{\kb}{m o m}{
    \IfNoValueTF{#2}{| #1 \rangle \langle #3 |}{| #1 \rangle #2 \langle #3 |}
}


% снимаю шляпу
% \renewcommand{\hat}[1]{#1}