\section{Диффузное расстояние}

% Мы предполагаем, что случайных количество шагов $K$ необходимое для достижения вершины монотонно связано с расстоянием 
Рассмотрим следующее семейство случайных блужданий: начинаем из стартовой вершины $V_0$, делаем $K \in [1, \sub{K}{max}]$ шагов $k$ блуждающими, в итоге получаем $k \sub{K}{max}$ вершин размеченных некоторым значением $K$. Конечно же в одну и ту же вершину можно прийти разными случайными блужданиями разной длительности, однако интересно посмотреть на зависимость \textit{диффузного расстояния} $D(V)$:
\begin{equation*}
    D(V) = \sum_{K=1}^{\sub{K}{max}} K \tilde{p}_K(V),
    \hspace{5 mm} 
    \tilde{\vc{p}} = \vc{p} / \|\vc{p}\|_1,
\end{equation*}
где $\vc{p} = (p_1(V), p_2(V), \ldots)$ -- вектор из вероятностей находиться в вершине $V$ на шаге $K$. Изначально нормировка такая, что $\sum_V p_K(V) = 1$.


Стоит заметить, что $D(V)$ зависит от выбора $\sub{K}{max}$. Более того, при $\sub{K}{max} \to \infty$ мы увидим расходимость $D(V)$ для конечных графов. Это легко увидеть, если вспомнить, что $p_{\infty}(V) \sim 1/N$, где $N$ это размер графа. Тогда в сумме можем оценить $\vc{p}_{\infty}(V) \sim 1/\sub{K}{max}$ и
\begin{equation*}
    \lim_{\sub{K}{max} \to \infty} D(V) \sim \sub{K}{max} / 2 \to \infty.
\end{equation*}
Где наступает этот момент, когда $D(V)$ теряет информативность? Пока не могу явно оценить, но например для кубика 222 можно заметить, когда $p_K(V_0)$ выходит на константу по $K$ (fig. \ref{fig:2}). 



\begin{figure}[h]
    \centering
    \includegraphics[width=0.8\linewidth]{C:/Users/khoru/Downloads/RC_2.pdf}
    \caption{Вероятность оказаться в $V_0$ после $K$ случайных шагов, каждая точка получена по $k=10^{8}$ блуждающим}
    \label{fig:2}
\end{figure}



Посмотрим для меньших значений на зависимость $D(d)$, где $d$ это истинная дистанция до $V_0$ (fig. \ref{fig:1}). Для удобства переопределим $D(V_0) \overset{\mathrm{def}}{=} 0$. В данном случае диаметр кубика равен $14$. Видно, что для $\sub{K}{max}$ сильно превыщающем диаметр (однако при $p_K(V) \neq \const$) диффузное расстояние остаётся информативным и монотонно зависит от $K$ (за исключением последних двух колец $d=13,14$, что не принципиально во время сборки кубика). 



\begin{figure}[h]
    \centering
    \includegraphics[width=0.8\linewidth]{C:/Users/khoru/Downloads/RC_1.pdf}
    \caption{Зависимость диффузного расстояния $D$ от истинного $d$, каждая точка получена по $k=10^{9}$ блуждающим}
    \label{fig:1}
\end{figure}



\begin{table}
\caption{Количество верщин в колце $d$ для кубика 222, QTM}
\begin{tabular}{cl}
\toprule
$d$ & counts \\
\midrule
0  & 1 \\
1  & 6 \\
2  & 27 \\
3  & 120 \\
4  & 534 \\
5  & 2256 \\
6  & 8969 \\
7  & 33058 \\
8  & 114149 \\
9  & 360508 \\
10 & 930588 \\
11 & 1350852 \\
12 & 782536 \\
13 & 90280 \\
14 & 276 \\
\bottomrule
\end{tabular}
\end{table}


% Lorem ipsum dolor sit amet, consectetur adipisicing elit, sed do eiusmod
% tempor incididunt ut labore et dolore magna aliqua. Ut enim ad minim veniam,
% quis nostrud exercitation ullamco laboris nisi ut aliquip ex ea commodo
% consequat. Duis aute irure dolor in reprehenderit in voluptate velit esse
% cillum dolore eu fugiat nulla pariatur. Excepteur sint occaecat cupidatat non
% proident, sunt in culpa qui officia deserunt mollit anim id est laborum.

\newpage

\phantom{42}

\newpage

\phantom{42}